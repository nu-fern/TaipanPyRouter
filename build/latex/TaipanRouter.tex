%% Generated by Sphinx.
\def\sphinxdocclass{report}
\documentclass[letterpaper,10pt,english]{sphinxmanual}
\ifdefined\pdfpxdimen
   \let\sphinxpxdimen\pdfpxdimen\else\newdimen\sphinxpxdimen
\fi \sphinxpxdimen=.75bp\relax

\PassOptionsToPackage{warn}{textcomp}
\usepackage[utf8]{inputenc}
\ifdefined\DeclareUnicodeCharacter
 \ifdefined\DeclareUnicodeCharacterAsOptional
  \DeclareUnicodeCharacter{"00A0}{\nobreakspace}
  \DeclareUnicodeCharacter{"2500}{\sphinxunichar{2500}}
  \DeclareUnicodeCharacter{"2502}{\sphinxunichar{2502}}
  \DeclareUnicodeCharacter{"2514}{\sphinxunichar{2514}}
  \DeclareUnicodeCharacter{"251C}{\sphinxunichar{251C}}
  \DeclareUnicodeCharacter{"2572}{\textbackslash}
 \else
  \DeclareUnicodeCharacter{00A0}{\nobreakspace}
  \DeclareUnicodeCharacter{2500}{\sphinxunichar{2500}}
  \DeclareUnicodeCharacter{2502}{\sphinxunichar{2502}}
  \DeclareUnicodeCharacter{2514}{\sphinxunichar{2514}}
  \DeclareUnicodeCharacter{251C}{\sphinxunichar{251C}}
  \DeclareUnicodeCharacter{2572}{\textbackslash}
 \fi
\fi
\usepackage{cmap}
\usepackage[T1]{fontenc}
\usepackage{amsmath,amssymb,amstext}
\usepackage{babel}
\usepackage{times}
\usepackage[Bjarne]{fncychap}
\usepackage{sphinx}

\usepackage{geometry}

% Include hyperref last.
\usepackage{hyperref}
% Fix anchor placement for figures with captions.
\usepackage{hypcap}% it must be loaded after hyperref.
% Set up styles of URL: it should be placed after hyperref.
\urlstyle{same}

\addto\captionsenglish{\renewcommand{\figurename}{Fig.}}
\addto\captionsenglish{\renewcommand{\tablename}{Table}}
\addto\captionsenglish{\renewcommand{\literalblockname}{Listing}}

\addto\captionsenglish{\renewcommand{\literalblockcontinuedname}{continued from previous page}}
\addto\captionsenglish{\renewcommand{\literalblockcontinuesname}{continues on next page}}

\addto\extrasenglish{\def\pageautorefname{page}}

\setcounter{tocdepth}{2}



\title{Taipan Router Documentation}
\date{Aug 07, 2018}
\release{}
\author{Carlos Bacigalupo}
\newcommand{\sphinxlogo}{\vbox{}}
\renewcommand{\releasename}{}
\makeindex

\begin{document}

\maketitle
\sphinxtableofcontents
\phantomsection\label{\detokenize{index::doc}}



\chapter{TAIPAN Router Manual}
\label{\detokenize{manual:taipan-router-documentation}}\label{\detokenize{manual:taipan-router-manual}}\label{\detokenize{manual::doc}}
The TAIPAN router calculates the paths that the starbugs need to take to be
positioned in a given arrangement.

The starting coordinates are the individual park positions as provided by the
locationProperties.json (S5). This file also contains the information on bug
availability allowing the identification of missing bugs.

The target coordinates are specified in the XY tiles (S2). These tiles are
created from the Tiles (S1) that are generated by the tiler code.

The step to convert S1 -\textgreater{} S2 is done by an external program creatively named S12S2.
This program uses the telescope model to convert RA/Dec coordinates to X/Y
positions in the focal plane.

The output of the code is a Routed Tile (S3) with the necessary steps to move the bugs
into position without colliding.

\noindent{\hspace*{\fill}\sphinxincludegraphics[width=800\sphinxpxdimen]{{GFP}.png}\hspace*{\fill}}


\chapter{TAIPAN Router reference manual}
\label{\detokenize{reference:taipan-router-reference-manual}}\label{\detokenize{reference:module-taipanPyRouter}}\label{\detokenize{reference::doc}}\index{taipanPyRouter (module)}\index{CGroupSolver (class in taipanPyRouter)}

\begin{fulllineitems}
\phantomsection\label{\detokenize{reference:taipanPyRouter.CGroupSolver}}\pysiglinewithargsret{\sphinxbfcode{\sphinxupquote{class }}\sphinxcode{\sphinxupquote{taipanPyRouter.}}\sphinxbfcode{\sphinxupquote{CGroupSolver}}}{\emph{CGroup}, \emph{tickArray}, \emph{bugStatus}}{}
Class to hold the crossing group solving funcitons.
\index{calcCoeffs() (taipanPyRouter.CGroupSolver method)}

\begin{fulllineitems}
\phantomsection\label{\detokenize{reference:taipanPyRouter.CGroupSolver.calcCoeffs}}\pysiglinewithargsret{\sphinxbfcode{\sphinxupquote{calcCoeffs}}}{\emph{A}, \emph{B}}{}
Calculates the collision coefficients.
\begin{itemize}
\item {} 
Given 2 ESegments, it returns the coefficients of the collision points.

\item {} 
If the ESegments are parallel, it returns NaNs

\end{itemize}
\begin{quote}\begin{description}
\item[{Parameters}] \leavevmode\begin{itemize}
\item {} 
\sphinxstyleliteralstrong{\sphinxupquote{A}} ({\hyperref[\detokenize{reference:taipanPyRouter.ESegment}]{\sphinxcrossref{\sphinxstyleliteralemphasis{\sphinxupquote{ESegment}}}}}) \textendash{} First segment to compare.

\item {} 
\sphinxstyleliteralstrong{\sphinxupquote{B}} ({\hyperref[\detokenize{reference:taipanPyRouter.ESegment}]{\sphinxcrossref{\sphinxstyleliteralemphasis{\sphinxupquote{ESegment}}}}}) \textendash{} Second segment to compare.

\end{itemize}

\item[{Returns}] \leavevmode
Pair of coefficients

\item[{Return type}] \leavevmode
Tuple

\end{description}\end{quote}

\end{fulllineitems}

\index{constructCMatrix() (taipanPyRouter.CGroupSolver method)}

\begin{fulllineitems}
\phantomsection\label{\detokenize{reference:taipanPyRouter.CGroupSolver.constructCMatrix}}\pysiglinewithargsret{\sphinxbfcode{\sphinxupquote{constructCMatrix}}}{}{}
Creates a distance matrix for all segments in the crossing group

\end{fulllineitems}

\index{constructESegments() (taipanPyRouter.CGroupSolver method)}

\begin{fulllineitems}
\phantomsection\label{\detokenize{reference:taipanPyRouter.CGroupSolver.constructESegments}}\pysiglinewithargsret{\sphinxbfcode{\sphinxupquote{constructESegments}}}{\emph{XYs}}{}
Constructs a list of all ESegments in the CGroup.
\begin{quote}\begin{description}
\item[{Parameters}] \leavevmode
\sphinxstyleliteralstrong{\sphinxupquote{XYs}} (\sphinxstyleliteralemphasis{\sphinxupquote{np.ndarray}}) \textendash{} List of coordinates of the end points of each ESegment

\item[{Returns}] \leavevmode
List of ESegments

\item[{Return type}] \leavevmode
list

\end{description}\end{quote}

\end{fulllineitems}

\index{findMovingSequence() (taipanPyRouter.CGroupSolver method)}

\begin{fulllineitems}
\phantomsection\label{\detokenize{reference:taipanPyRouter.CGroupSolver.findMovingSequence}}\pysiglinewithargsret{\sphinxbfcode{\sphinxupquote{findMovingSequence}}}{}{}
Analyses the cost matrix to find the sequence of motion that doesn’t crash.

\end{fulllineitems}

\index{findMovingSequenceBF() (taipanPyRouter.CGroupSolver method)}

\begin{fulllineitems}
\phantomsection\label{\detokenize{reference:taipanPyRouter.CGroupSolver.findMovingSequenceBF}}\pysiglinewithargsret{\sphinxbfcode{\sphinxupquote{findMovingSequenceBF}}}{}{}
\end{fulllineitems}


\end{fulllineitems}

\index{ESegment (class in taipanPyRouter)}

\begin{fulllineitems}
\phantomsection\label{\detokenize{reference:taipanPyRouter.ESegment}}\pysiglinewithargsret{\sphinxbfcode{\sphinxupquote{class }}\sphinxcode{\sphinxupquote{taipanPyRouter.}}\sphinxbfcode{\sphinxupquote{ESegment}}}{\emph{XYs}}{}
Bases: \sphinxcode{\sphinxupquote{shapely.geometry.linestring.LineString}}

Class to extend the existing Linestring class into E(xtended)Segments.

ESegments provide extra elements that apply to route solving
\index{almost\_equals() (taipanPyRouter.ESegment method)}

\begin{fulllineitems}
\phantomsection\label{\detokenize{reference:taipanPyRouter.ESegment.almost_equals}}\pysiglinewithargsret{\sphinxbfcode{\sphinxupquote{almost\_equals}}}{\emph{other}, \emph{decimal=6}}{}
Returns True if geometries are equal at all coordinates to a
specified decimal place

Refers to approximate coordinate equality, which requires coordinates be
approximately equal and in the same order for all components of a geometry.

\end{fulllineitems}

\index{area (taipanPyRouter.ESegment attribute)}

\begin{fulllineitems}
\phantomsection\label{\detokenize{reference:taipanPyRouter.ESegment.area}}\pysigline{\sphinxbfcode{\sphinxupquote{area}}}
Unitless area of the geometry (float)

\end{fulllineitems}

\index{array\_interface() (taipanPyRouter.ESegment method)}

\begin{fulllineitems}
\phantomsection\label{\detokenize{reference:taipanPyRouter.ESegment.array_interface}}\pysiglinewithargsret{\sphinxbfcode{\sphinxupquote{array\_interface}}}{}{}
Provide the Numpy array protocol.

\end{fulllineitems}

\index{array\_interface\_base (taipanPyRouter.ESegment attribute)}

\begin{fulllineitems}
\phantomsection\label{\detokenize{reference:taipanPyRouter.ESegment.array_interface_base}}\pysigline{\sphinxbfcode{\sphinxupquote{array\_interface\_base}}}
\end{fulllineitems}

\index{boundary (taipanPyRouter.ESegment attribute)}

\begin{fulllineitems}
\phantomsection\label{\detokenize{reference:taipanPyRouter.ESegment.boundary}}\pysigline{\sphinxbfcode{\sphinxupquote{boundary}}}
Returns a lower dimension geometry that bounds the object

The boundary of a polygon is a line, the boundary of a line is a
collection of points. The boundary of a point is an empty (null)
collection.

\end{fulllineitems}

\index{bounds (taipanPyRouter.ESegment attribute)}

\begin{fulllineitems}
\phantomsection\label{\detokenize{reference:taipanPyRouter.ESegment.bounds}}\pysigline{\sphinxbfcode{\sphinxupquote{bounds}}}
Returns minimum bounding region (minx, miny, maxx, maxy)

\end{fulllineitems}

\index{buffer() (taipanPyRouter.ESegment method)}

\begin{fulllineitems}
\phantomsection\label{\detokenize{reference:taipanPyRouter.ESegment.buffer}}\pysiglinewithargsret{\sphinxbfcode{\sphinxupquote{buffer}}}{\emph{distance}, \emph{resolution=16}, \emph{quadsegs=None}, \emph{cap\_style=1}, \emph{join\_style=1}, \emph{mitre\_limit=5.0}}{}
Returns a geometry with an envelope at a distance from the object’s
envelope

A negative distance has a “shrink” effect. A zero distance may be used
to “tidy” a polygon. The resolution of the buffer around each vertex of
the object increases by increasing the resolution keyword parameter
or second positional parameter. Note: the use of a \sphinxtitleref{quadsegs} parameter
is deprecated and will be gone from the next major release.

The styles of caps are: CAP\_STYLE.round (1), CAP\_STYLE.flat (2), and
CAP\_STYLE.square (3).

The styles of joins between offset segments are: JOIN\_STYLE.round (1),
JOIN\_STYLE.mitre (2), and JOIN\_STYLE.bevel (3).

The mitre limit ratio is used for very sharp corners. The mitre ratio
is the ratio of the distance from the corner to the end of the mitred
offset corner. When two line segments meet at a sharp angle, a miter
join will extend the original geometry. To prevent unreasonable
geometry, the mitre limit allows controlling the maximum length of the
join corner. Corners with a ratio which exceed the limit will be
beveled.
\paragraph{Example}

\fvset{hllines={, ,}}%
\begin{sphinxVerbatim}[commandchars=\\\{\}]
\PYG{g+gp}{\PYGZgt{}\PYGZgt{}\PYGZgt{} }\PYG{k+kn}{from} \PYG{n+nn}{shapely}\PYG{n+nn}{.}\PYG{n+nn}{wkt} \PYG{k}{import} \PYG{n}{loads}
\PYG{g+gp}{\PYGZgt{}\PYGZgt{}\PYGZgt{} }\PYG{n}{g} \PYG{o}{=} \PYG{n}{loads}\PYG{p}{(}\PYG{l+s+s1}{\PYGZsq{}}\PYG{l+s+s1}{POINT (0.0 0.0)}\PYG{l+s+s1}{\PYGZsq{}}\PYG{p}{)}
\PYG{g+gp}{\PYGZgt{}\PYGZgt{}\PYGZgt{} }\PYG{n}{g}\PYG{o}{.}\PYG{n}{buffer}\PYG{p}{(}\PYG{l+m+mf}{1.0}\PYG{p}{)}\PYG{o}{.}\PYG{n}{area}        \PYG{c+c1}{\PYGZsh{} 16\PYGZhy{}gon approx of a unit radius circle}
\PYG{g+go}{3.1365484905459389}
\PYG{g+gp}{\PYGZgt{}\PYGZgt{}\PYGZgt{} }\PYG{n}{g}\PYG{o}{.}\PYG{n}{buffer}\PYG{p}{(}\PYG{l+m+mf}{1.0}\PYG{p}{,} \PYG{l+m+mi}{128}\PYG{p}{)}\PYG{o}{.}\PYG{n}{area}   \PYG{c+c1}{\PYGZsh{} 128\PYGZhy{}gon approximation}
\PYG{g+go}{3.1415138011443009}
\PYG{g+gp}{\PYGZgt{}\PYGZgt{}\PYGZgt{} }\PYG{n}{g}\PYG{o}{.}\PYG{n}{buffer}\PYG{p}{(}\PYG{l+m+mf}{1.0}\PYG{p}{,} \PYG{l+m+mi}{3}\PYG{p}{)}\PYG{o}{.}\PYG{n}{area}     \PYG{c+c1}{\PYGZsh{} triangle approximation}
\PYG{g+go}{3.0}
\PYG{g+gp}{\PYGZgt{}\PYGZgt{}\PYGZgt{} }\PYG{n+nb}{list}\PYG{p}{(}\PYG{n}{g}\PYG{o}{.}\PYG{n}{buffer}\PYG{p}{(}\PYG{l+m+mf}{1.0}\PYG{p}{,} \PYG{n}{cap\PYGZus{}style}\PYG{o}{=}\PYG{l+s+s1}{\PYGZsq{}}\PYG{l+s+s1}{square}\PYG{l+s+s1}{\PYGZsq{}}\PYG{p}{)}\PYG{o}{.}\PYG{n}{exterior}\PYG{o}{.}\PYG{n}{coords}\PYG{p}{)}
\PYG{g+go}{[(1.0, 1.0), (1.0, \PYGZhy{}1.0), (\PYGZhy{}1.0, \PYGZhy{}1.0), (\PYGZhy{}1.0, 1.0), (1.0, 1.0)]}
\PYG{g+gp}{\PYGZgt{}\PYGZgt{}\PYGZgt{} }\PYG{n}{g}\PYG{o}{.}\PYG{n}{buffer}\PYG{p}{(}\PYG{l+m+mf}{1.0}\PYG{p}{,} \PYG{n}{cap\PYGZus{}style}\PYG{o}{=}\PYG{l+s+s1}{\PYGZsq{}}\PYG{l+s+s1}{square}\PYG{l+s+s1}{\PYGZsq{}}\PYG{p}{)}\PYG{o}{.}\PYG{n}{area}
\PYG{g+go}{4.0}
\end{sphinxVerbatim}

\end{fulllineitems}

\index{centroid (taipanPyRouter.ESegment attribute)}

\begin{fulllineitems}
\phantomsection\label{\detokenize{reference:taipanPyRouter.ESegment.centroid}}\pysigline{\sphinxbfcode{\sphinxupquote{centroid}}}
Returns the geometric center of the object

\end{fulllineitems}

\index{contains() (taipanPyRouter.ESegment method)}

\begin{fulllineitems}
\phantomsection\label{\detokenize{reference:taipanPyRouter.ESegment.contains}}\pysiglinewithargsret{\sphinxbfcode{\sphinxupquote{contains}}}{\emph{other}}{}
Returns True if the geometry contains the other, else False

\end{fulllineitems}

\index{convex\_hull (taipanPyRouter.ESegment attribute)}

\begin{fulllineitems}
\phantomsection\label{\detokenize{reference:taipanPyRouter.ESegment.convex_hull}}\pysigline{\sphinxbfcode{\sphinxupquote{convex\_hull}}}
\sphinxstyleemphasis{Imagine an elastic band stretched around the geometry} \textendash{} that’s a
convex hull, more or less

The convex hull of a three member multipoint, for example, is a
triangular polygon.

\end{fulllineitems}

\index{coords (taipanPyRouter.ESegment attribute)}

\begin{fulllineitems}
\phantomsection\label{\detokenize{reference:taipanPyRouter.ESegment.coords}}\pysigline{\sphinxbfcode{\sphinxupquote{coords}}}
Access to geometry’s coordinates (CoordinateSequence)

\end{fulllineitems}

\index{covers() (taipanPyRouter.ESegment method)}

\begin{fulllineitems}
\phantomsection\label{\detokenize{reference:taipanPyRouter.ESegment.covers}}\pysiglinewithargsret{\sphinxbfcode{\sphinxupquote{covers}}}{\emph{other}}{}
Returns True if the geometry covers the other, else False

\end{fulllineitems}

\index{crosses() (taipanPyRouter.ESegment method)}

\begin{fulllineitems}
\phantomsection\label{\detokenize{reference:taipanPyRouter.ESegment.crosses}}\pysiglinewithargsret{\sphinxbfcode{\sphinxupquote{crosses}}}{\emph{other}}{}
Returns True if the geometries cross, else False

\end{fulllineitems}

\index{ctypes (taipanPyRouter.ESegment attribute)}

\begin{fulllineitems}
\phantomsection\label{\detokenize{reference:taipanPyRouter.ESegment.ctypes}}\pysigline{\sphinxbfcode{\sphinxupquote{ctypes}}}
\end{fulllineitems}

\index{difference() (taipanPyRouter.ESegment method)}

\begin{fulllineitems}
\phantomsection\label{\detokenize{reference:taipanPyRouter.ESegment.difference}}\pysiglinewithargsret{\sphinxbfcode{\sphinxupquote{difference}}}{\emph{other}}{}
Returns the difference of the geometries

\end{fulllineitems}

\index{disjoint() (taipanPyRouter.ESegment method)}

\begin{fulllineitems}
\phantomsection\label{\detokenize{reference:taipanPyRouter.ESegment.disjoint}}\pysiglinewithargsret{\sphinxbfcode{\sphinxupquote{disjoint}}}{\emph{other}}{}
Returns True if geometries are disjoint, else False

\end{fulllineitems}

\index{distance() (taipanPyRouter.ESegment method)}

\begin{fulllineitems}
\phantomsection\label{\detokenize{reference:taipanPyRouter.ESegment.distance}}\pysiglinewithargsret{\sphinxbfcode{\sphinxupquote{distance}}}{\emph{other}}{}
Unitless distance to other geometry (float)

\end{fulllineitems}

\index{empty() (taipanPyRouter.ESegment method)}

\begin{fulllineitems}
\phantomsection\label{\detokenize{reference:taipanPyRouter.ESegment.empty}}\pysiglinewithargsret{\sphinxbfcode{\sphinxupquote{empty}}}{\emph{val=4460832448}}{}
\end{fulllineitems}

\index{envelope (taipanPyRouter.ESegment attribute)}

\begin{fulllineitems}
\phantomsection\label{\detokenize{reference:taipanPyRouter.ESegment.envelope}}\pysigline{\sphinxbfcode{\sphinxupquote{envelope}}}
A figure that envelopes the geometry

\end{fulllineitems}

\index{equals() (taipanPyRouter.ESegment method)}

\begin{fulllineitems}
\phantomsection\label{\detokenize{reference:taipanPyRouter.ESegment.equals}}\pysiglinewithargsret{\sphinxbfcode{\sphinxupquote{equals}}}{\emph{other}}{}
Returns True if geometries are equal, else False

Refers to point-set equality (or topological equality), and is equivalent to
(self.within(other) \& self.contains(other))

\end{fulllineitems}

\index{equals\_exact() (taipanPyRouter.ESegment method)}

\begin{fulllineitems}
\phantomsection\label{\detokenize{reference:taipanPyRouter.ESegment.equals_exact}}\pysiglinewithargsret{\sphinxbfcode{\sphinxupquote{equals\_exact}}}{\emph{other}, \emph{tolerance}}{}
Returns True if geometries are equal to within a specified
tolerance

Refers to coordinate equality, which requires coordinates to be equal
and in the same order for all components of a geometry

\end{fulllineitems}

\index{geom\_type (taipanPyRouter.ESegment attribute)}

\begin{fulllineitems}
\phantomsection\label{\detokenize{reference:taipanPyRouter.ESegment.geom_type}}\pysigline{\sphinxbfcode{\sphinxupquote{geom\_type}}}
Name of the geometry’s type, such as ‘Point’

\end{fulllineitems}

\index{geometryType() (taipanPyRouter.ESegment method)}

\begin{fulllineitems}
\phantomsection\label{\detokenize{reference:taipanPyRouter.ESegment.geometryType}}\pysiglinewithargsret{\sphinxbfcode{\sphinxupquote{geometryType}}}{}{}
\end{fulllineitems}

\index{has\_z (taipanPyRouter.ESegment attribute)}

\begin{fulllineitems}
\phantomsection\label{\detokenize{reference:taipanPyRouter.ESegment.has_z}}\pysigline{\sphinxbfcode{\sphinxupquote{has\_z}}}
True if the geometry’s coordinate sequence(s) have z values (are
3-dimensional)

\end{fulllineitems}

\index{hausdorff\_distance() (taipanPyRouter.ESegment method)}

\begin{fulllineitems}
\phantomsection\label{\detokenize{reference:taipanPyRouter.ESegment.hausdorff_distance}}\pysiglinewithargsret{\sphinxbfcode{\sphinxupquote{hausdorff\_distance}}}{\emph{other}}{}
Unitless hausdorff distance to other geometry (float)

\end{fulllineitems}

\index{impl (taipanPyRouter.ESegment attribute)}

\begin{fulllineitems}
\phantomsection\label{\detokenize{reference:taipanPyRouter.ESegment.impl}}\pysigline{\sphinxbfcode{\sphinxupquote{impl}}\sphinxbfcode{\sphinxupquote{ = \textless{}GEOSImpl object: GEOS C API version (1, 8, 0)\textgreater{}}}}
\end{fulllineitems}

\index{interpolate() (taipanPyRouter.ESegment method)}

\begin{fulllineitems}
\phantomsection\label{\detokenize{reference:taipanPyRouter.ESegment.interpolate}}\pysiglinewithargsret{\sphinxbfcode{\sphinxupquote{interpolate}}}{\emph{**kwargs}}{}
Return a point at the specified distance along a linear geometry

If the normalized arg is True, the distance will be interpreted as a
fraction of the geometry’s length.

\end{fulllineitems}

\index{intersection() (taipanPyRouter.ESegment method)}

\begin{fulllineitems}
\phantomsection\label{\detokenize{reference:taipanPyRouter.ESegment.intersection}}\pysiglinewithargsret{\sphinxbfcode{\sphinxupquote{intersection}}}{\emph{other}}{}
Returns the intersection of the geometries

\end{fulllineitems}

\index{intersects() (taipanPyRouter.ESegment method)}

\begin{fulllineitems}
\phantomsection\label{\detokenize{reference:taipanPyRouter.ESegment.intersects}}\pysiglinewithargsret{\sphinxbfcode{\sphinxupquote{intersects}}}{\emph{other}}{}
Returns True if geometries intersect, else False

\end{fulllineitems}

\index{is\_closed (taipanPyRouter.ESegment attribute)}

\begin{fulllineitems}
\phantomsection\label{\detokenize{reference:taipanPyRouter.ESegment.is_closed}}\pysigline{\sphinxbfcode{\sphinxupquote{is\_closed}}}
True if the geometry is closed, else False

Applicable only to 1-D geometries.

\end{fulllineitems}

\index{is\_empty (taipanPyRouter.ESegment attribute)}

\begin{fulllineitems}
\phantomsection\label{\detokenize{reference:taipanPyRouter.ESegment.is_empty}}\pysigline{\sphinxbfcode{\sphinxupquote{is\_empty}}}
True if the set of points in this geometry is empty, else False

\end{fulllineitems}

\index{is\_ring (taipanPyRouter.ESegment attribute)}

\begin{fulllineitems}
\phantomsection\label{\detokenize{reference:taipanPyRouter.ESegment.is_ring}}\pysigline{\sphinxbfcode{\sphinxupquote{is\_ring}}}
True if the geometry is a closed ring, else False

\end{fulllineitems}

\index{is\_simple (taipanPyRouter.ESegment attribute)}

\begin{fulllineitems}
\phantomsection\label{\detokenize{reference:taipanPyRouter.ESegment.is_simple}}\pysigline{\sphinxbfcode{\sphinxupquote{is\_simple}}}
True if the geometry is simple, meaning that any self-intersections
are only at boundary points, else False

\end{fulllineitems}

\index{is\_valid (taipanPyRouter.ESegment attribute)}

\begin{fulllineitems}
\phantomsection\label{\detokenize{reference:taipanPyRouter.ESegment.is_valid}}\pysigline{\sphinxbfcode{\sphinxupquote{is\_valid}}}
True if the geometry is valid (definition depends on sub-class),
else False

\end{fulllineitems}

\index{length (taipanPyRouter.ESegment attribute)}

\begin{fulllineitems}
\phantomsection\label{\detokenize{reference:taipanPyRouter.ESegment.length}}\pysigline{\sphinxbfcode{\sphinxupquote{length}}}
Unitless length of the geometry (float)

\end{fulllineitems}

\index{minimum\_rotated\_rectangle (taipanPyRouter.ESegment attribute)}

\begin{fulllineitems}
\phantomsection\label{\detokenize{reference:taipanPyRouter.ESegment.minimum_rotated_rectangle}}\pysigline{\sphinxbfcode{\sphinxupquote{minimum\_rotated\_rectangle}}}
Returns the general minimum bounding rectangle of
the geometry. Can possibly be rotated. If the convex hull
of the object is a degenerate (line or point) this same degenerate
is returned.

\end{fulllineitems}

\index{overlaps() (taipanPyRouter.ESegment method)}

\begin{fulllineitems}
\phantomsection\label{\detokenize{reference:taipanPyRouter.ESegment.overlaps}}\pysiglinewithargsret{\sphinxbfcode{\sphinxupquote{overlaps}}}{\emph{other}}{}
Returns True if geometries overlap, else False

\end{fulllineitems}

\index{parallel\_offset() (taipanPyRouter.ESegment method)}

\begin{fulllineitems}
\phantomsection\label{\detokenize{reference:taipanPyRouter.ESegment.parallel_offset}}\pysiglinewithargsret{\sphinxbfcode{\sphinxupquote{parallel\_offset}}}{\emph{distance}, \emph{side='right'}, \emph{resolution=16}, \emph{join\_style=1}, \emph{mitre\_limit=5.0}}{}
Returns a LineString or MultiLineString geometry at a distance from
the object on its right or its left side.

The side parameter may be ‘left’ or ‘right’ (default is ‘right’). The
resolution of the buffer around each vertex of the object increases by
increasing the resolution keyword parameter or third positional
parameter. Vertices of right hand offset lines will be ordered in
reverse.

The join style is for outside corners between line segments. Accepted
values are JOIN\_STYLE.round (1), JOIN\_STYLE.mitre (2), and
JOIN\_STYLE.bevel (3).

The mitre ratio limit is used for very sharp corners. It is the ratio
of the distance from the corner to the end of the mitred offset corner.
When two line segments meet at a sharp angle, a miter join will extend
far beyond the original geometry. To prevent unreasonable geometry, the
mitre limit allows controlling the maximum length of the join corner.
Corners with a ratio which exceed the limit will be beveled.

\end{fulllineitems}

\index{pointFromCoeff() (taipanPyRouter.ESegment method)}

\begin{fulllineitems}
\phantomsection\label{\detokenize{reference:taipanPyRouter.ESegment.pointFromCoeff}}\pysiglinewithargsret{\sphinxbfcode{\sphinxupquote{pointFromCoeff}}}{\emph{coeff}}{}
Calculates a projected point from a given coefficient.

The resulting point is in the position coeff*length, where length is
the length of the ESegment.
\begin{quote}\begin{description}
\item[{Parameters}] \leavevmode
\sphinxstyleliteralstrong{\sphinxupquote{coeff}} (\sphinxstyleliteralemphasis{\sphinxupquote{float}}) \textendash{} Distance to the requested point in ESegment lengths. Can be negative.

\item[{Returns}] \leavevmode
Position of the point.

\item[{Return type}] \leavevmode
Tuple

\end{description}\end{quote}

\end{fulllineitems}

\index{project() (taipanPyRouter.ESegment method)}

\begin{fulllineitems}
\phantomsection\label{\detokenize{reference:taipanPyRouter.ESegment.project}}\pysiglinewithargsret{\sphinxbfcode{\sphinxupquote{project}}}{\emph{**kwargs}}{}
Returns the distance along this geometry to a point nearest the
specified point

If the normalized arg is True, return the distance normalized to the
length of the linear geometry.

\end{fulllineitems}

\index{relate() (taipanPyRouter.ESegment method)}

\begin{fulllineitems}
\phantomsection\label{\detokenize{reference:taipanPyRouter.ESegment.relate}}\pysiglinewithargsret{\sphinxbfcode{\sphinxupquote{relate}}}{\emph{other}}{}
Returns the DE-9IM intersection matrix for the two geometries
(string)

\end{fulllineitems}

\index{relate\_pattern() (taipanPyRouter.ESegment method)}

\begin{fulllineitems}
\phantomsection\label{\detokenize{reference:taipanPyRouter.ESegment.relate_pattern}}\pysiglinewithargsret{\sphinxbfcode{\sphinxupquote{relate\_pattern}}}{\emph{other}, \emph{pattern}}{}
Returns True if the DE-9IM string code for the relationship between
the geometries satisfies the pattern, else False

\end{fulllineitems}

\index{representative\_point() (taipanPyRouter.ESegment method)}

\begin{fulllineitems}
\phantomsection\label{\detokenize{reference:taipanPyRouter.ESegment.representative_point}}\pysiglinewithargsret{\sphinxbfcode{\sphinxupquote{representative\_point}}}{\emph{**kwargs}}{}
Returns a point guaranteed to be within the object, cheaply.

\end{fulllineitems}

\index{simplify() (taipanPyRouter.ESegment method)}

\begin{fulllineitems}
\phantomsection\label{\detokenize{reference:taipanPyRouter.ESegment.simplify}}\pysiglinewithargsret{\sphinxbfcode{\sphinxupquote{simplify}}}{\emph{**kwargs}}{}
Returns a simplified geometry produced by the Douglas-Peucker
algorithm

Coordinates of the simplified geometry will be no more than the
tolerance distance from the original. Unless the topology preserving
option is used, the algorithm may produce self-intersecting or
otherwise invalid geometries.

\end{fulllineitems}

\index{svg() (taipanPyRouter.ESegment method)}

\begin{fulllineitems}
\phantomsection\label{\detokenize{reference:taipanPyRouter.ESegment.svg}}\pysiglinewithargsret{\sphinxbfcode{\sphinxupquote{svg}}}{\emph{scale\_factor=1.0}, \emph{stroke\_color=None}}{}
Returns SVG polyline element for the LineString geometry.
\begin{quote}\begin{description}
\item[{Parameters}] \leavevmode\begin{itemize}
\item {} 
\sphinxstyleliteralstrong{\sphinxupquote{scale\_factor}} (\sphinxstyleliteralemphasis{\sphinxupquote{float}}) \textendash{} Multiplication factor for the SVG stroke-width.  Default is 1.

\item {} 
\sphinxstyleliteralstrong{\sphinxupquote{stroke\_color}} (\sphinxstyleliteralemphasis{\sphinxupquote{str}}\sphinxstyleliteralemphasis{\sphinxupquote{, }}\sphinxstyleliteralemphasis{\sphinxupquote{optional}}) \textendash{} Hex string for stroke color. Default is to use “\#66cc99” if
geometry is valid, and “\#ff3333” if invalid.

\end{itemize}

\end{description}\end{quote}

\end{fulllineitems}

\index{symmetric\_difference() (taipanPyRouter.ESegment method)}

\begin{fulllineitems}
\phantomsection\label{\detokenize{reference:taipanPyRouter.ESegment.symmetric_difference}}\pysiglinewithargsret{\sphinxbfcode{\sphinxupquote{symmetric\_difference}}}{\emph{other}}{}
Returns the symmetric difference of the geometries
(Shapely geometry)

\end{fulllineitems}

\index{to\_wkb() (taipanPyRouter.ESegment method)}

\begin{fulllineitems}
\phantomsection\label{\detokenize{reference:taipanPyRouter.ESegment.to_wkb}}\pysiglinewithargsret{\sphinxbfcode{\sphinxupquote{to\_wkb}}}{}{}
\end{fulllineitems}

\index{to\_wkt() (taipanPyRouter.ESegment method)}

\begin{fulllineitems}
\phantomsection\label{\detokenize{reference:taipanPyRouter.ESegment.to_wkt}}\pysiglinewithargsret{\sphinxbfcode{\sphinxupquote{to\_wkt}}}{}{}
\end{fulllineitems}

\index{touches() (taipanPyRouter.ESegment method)}

\begin{fulllineitems}
\phantomsection\label{\detokenize{reference:taipanPyRouter.ESegment.touches}}\pysiglinewithargsret{\sphinxbfcode{\sphinxupquote{touches}}}{\emph{other}}{}
Returns True if geometries touch, else False

\end{fulllineitems}

\index{type (taipanPyRouter.ESegment attribute)}

\begin{fulllineitems}
\phantomsection\label{\detokenize{reference:taipanPyRouter.ESegment.type}}\pysigline{\sphinxbfcode{\sphinxupquote{type}}}
\end{fulllineitems}

\index{union() (taipanPyRouter.ESegment method)}

\begin{fulllineitems}
\phantomsection\label{\detokenize{reference:taipanPyRouter.ESegment.union}}\pysiglinewithargsret{\sphinxbfcode{\sphinxupquote{union}}}{\emph{other}}{}
Returns the union of the geometries (Shapely geometry)

\end{fulllineitems}

\index{within() (taipanPyRouter.ESegment method)}

\begin{fulllineitems}
\phantomsection\label{\detokenize{reference:taipanPyRouter.ESegment.within}}\pysiglinewithargsret{\sphinxbfcode{\sphinxupquote{within}}}{\emph{other}}{}
Returns True if geometry is within the other, else False

\end{fulllineitems}

\index{wkb (taipanPyRouter.ESegment attribute)}

\begin{fulllineitems}
\phantomsection\label{\detokenize{reference:taipanPyRouter.ESegment.wkb}}\pysigline{\sphinxbfcode{\sphinxupquote{wkb}}}
WKB representation of the geometry

\end{fulllineitems}

\index{wkb\_hex (taipanPyRouter.ESegment attribute)}

\begin{fulllineitems}
\phantomsection\label{\detokenize{reference:taipanPyRouter.ESegment.wkb_hex}}\pysigline{\sphinxbfcode{\sphinxupquote{wkb\_hex}}}
WKB hex representation of the geometry

\end{fulllineitems}

\index{wkt (taipanPyRouter.ESegment attribute)}

\begin{fulllineitems}
\phantomsection\label{\detokenize{reference:taipanPyRouter.ESegment.wkt}}\pysigline{\sphinxbfcode{\sphinxupquote{wkt}}}
WKT representation of the geometry

\end{fulllineitems}

\index{xy (taipanPyRouter.ESegment attribute)}

\begin{fulllineitems}
\phantomsection\label{\detokenize{reference:taipanPyRouter.ESegment.xy}}\pysigline{\sphinxbfcode{\sphinxupquote{xy}}}
Separate arrays of X and Y coordinate values
\paragraph{Example}

\fvset{hllines={, ,}}%
\begin{sphinxVerbatim}[commandchars=\\\{\}]
\PYG{g+gp}{\PYGZgt{}\PYGZgt{}\PYGZgt{} }\PYG{n}{x}\PYG{p}{,} \PYG{n}{y} \PYG{o}{=} \PYG{n}{LineString}\PYG{p}{(}\PYG{p}{(}\PYG{p}{(}\PYG{l+m+mi}{0}\PYG{p}{,} \PYG{l+m+mi}{0}\PYG{p}{)}\PYG{p}{,} \PYG{p}{(}\PYG{l+m+mi}{1}\PYG{p}{,} \PYG{l+m+mi}{1}\PYG{p}{)}\PYG{p}{)}\PYG{p}{)}\PYG{o}{.}\PYG{n}{xy}
\PYG{g+gp}{\PYGZgt{}\PYGZgt{}\PYGZgt{} }\PYG{n+nb}{list}\PYG{p}{(}\PYG{n}{x}\PYG{p}{)}
\PYG{g+go}{[0.0, 1.0]}
\PYG{g+gp}{\PYGZgt{}\PYGZgt{}\PYGZgt{} }\PYG{n+nb}{list}\PYG{p}{(}\PYG{n}{y}\PYG{p}{)}
\PYG{g+go}{[0.0, 1.0]}
\end{sphinxVerbatim}

\end{fulllineitems}


\end{fulllineitems}

\index{checkForCollisions() (in module taipanPyRouter)}

\begin{fulllineitems}
\phantomsection\label{\detokenize{reference:taipanPyRouter.checkForCollisions}}\pysiglinewithargsret{\sphinxcode{\sphinxupquote{taipanPyRouter.}}\sphinxbfcode{\sphinxupquote{checkForCollisions}}}{\emph{tickArray}, \emph{bugStatus}, \emph{booPrint=True}}{}
Looks for collisions in the created tick array.
\begin{itemize}
\item {} 
Steps through the tickArray 1 tick at the time.

\item {} 
Tries to re-create the crossing groups to look for crossings

\end{itemize}
\begin{quote}\begin{description}
\item[{Parameters}] \leavevmode\begin{itemize}
\item {} 
\sphinxstyleliteralstrong{\sphinxupquote{tickArray}} (\sphinxstyleliteralemphasis{\sphinxupquote{np.ndarray}}) \textendash{} Array with the routing solution

\item {} 
\sphinxstyleliteralstrong{\sphinxupquote{bugStatus}} (\sphinxstyleliteralemphasis{\sphinxupquote{np.ndarray}}) \textendash{} Status of the bugs

\end{itemize}

\item[{Returns}] \leavevmode
True if collisions found.

\item[{Return type}] \leavevmode
boolean

\end{description}\end{quote}

\end{fulllineitems}

\index{checkValidGFPandCrash() (in module taipanPyRouter)}

\begin{fulllineitems}
\phantomsection\label{\detokenize{reference:taipanPyRouter.checkValidGFPandCrash}}\pysiglinewithargsret{\sphinxcode{\sphinxupquote{taipanPyRouter.}}\sphinxbfcode{\sphinxupquote{checkValidGFPandCrash}}}{\emph{bugsTargetXY}, \emph{bugTargetTypes}}{}~\begin{description}
\item[{Checks that targets are:}] \leavevmode\begin{itemize}
\item {} 
inside GFP

\item {} 
far enough from eachother

\end{itemize}

\end{description}

Removes
\begin{quote}\begin{description}
\item[{Parameters}] \leavevmode\begin{itemize}
\item {} 
\sphinxstyleliteralstrong{\sphinxupquote{bugsTargetXY}} \textendash{} array with the target points

\item {} 
\sphinxstyleliteralstrong{\sphinxupquote{bugsTargetStatus}} \textendash{} array with the end points

\end{itemize}

\end{description}\end{quote}

\end{fulllineitems}

\index{checkValidTargets() (in module taipanPyRouter)}

\begin{fulllineitems}
\phantomsection\label{\detokenize{reference:taipanPyRouter.checkValidTargets}}\pysiglinewithargsret{\sphinxcode{\sphinxupquote{taipanPyRouter.}}\sphinxbfcode{\sphinxupquote{checkValidTargets}}}{\emph{parkPos}, \emph{bugsTargetXY}, \emph{bugStatus}, \emph{bugRouting}}{}~\begin{description}
\item[{Checks that targets are:}] \leavevmode\begin{itemize}
\item {} 
inside patrol radius

\item {} 
far from static bugs

\end{itemize}

\end{description}

Updates bugStatus accordingly
\begin{quote}\begin{description}
\item[{Parameters}] \leavevmode\begin{itemize}
\item {} 
\sphinxstyleliteralstrong{\sphinxupquote{parkPos}} \textendash{} array with the starting points

\item {} 
\sphinxstyleliteralstrong{\sphinxupquote{bugsTargetXY}} \textendash{} array with the end points

\end{itemize}

\end{description}\end{quote}

\end{fulllineitems}

\index{checkValidTargetsPR() (in module taipanPyRouter)}

\begin{fulllineitems}
\phantomsection\label{\detokenize{reference:taipanPyRouter.checkValidTargetsPR}}\pysiglinewithargsret{\sphinxcode{\sphinxupquote{taipanPyRouter.}}\sphinxbfcode{\sphinxupquote{checkValidTargetsPR}}}{\emph{parkPos}, \emph{bugsTargetXY}, \emph{bugStatus}, \emph{bugRouting}}{}~\begin{description}
\item[{Checks that targets are:}] \leavevmode\begin{itemize}
\item {} 
inside patrol radius

\item {} 
far from static bugs

\end{itemize}

\end{description}

Updates bugStatus accordingly
\begin{quote}\begin{description}
\item[{Parameters}] \leavevmode\begin{itemize}
\item {} 
\sphinxstyleliteralstrong{\sphinxupquote{parkPos}} \textendash{} array with the starting points

\item {} 
\sphinxstyleliteralstrong{\sphinxupquote{bugsTargetXY}} \textendash{} array with the end points

\end{itemize}

\end{description}\end{quote}

\end{fulllineitems}

\index{consolidateCGroups() (in module taipanPyRouter)}

\begin{fulllineitems}
\phantomsection\label{\detokenize{reference:taipanPyRouter.consolidateCGroups}}\pysiglinewithargsret{\sphinxcode{\sphinxupquote{taipanPyRouter.}}\sphinxbfcode{\sphinxupquote{consolidateCGroups}}}{\emph{crossingBugs}, \emph{tickArray}}{}
Takes all pairs that collide and groups them assigning a unique ID.
\begin{quote}\begin{description}
\item[{Parameters}] \leavevmode\begin{itemize}
\item {} 
\sphinxstyleliteralstrong{\sphinxupquote{crossingBugs}} \textendash{} np.ndarray Collection of pairs of crossing bugs.

\item {} 
\sphinxstyleliteralstrong{\sphinxupquote{tickArray}} \textendash{} np.ndarray {[}lemoID-1, tick, coords{]}

\end{itemize}

\end{description}\end{quote}

\end{fulllineitems}

\index{createWorkingFolder() (in module taipanPyRouter)}

\begin{fulllineitems}
\phantomsection\label{\detokenize{reference:taipanPyRouter.createWorkingFolder}}\pysiglinewithargsret{\sphinxcode{\sphinxupquote{taipanPyRouter.}}\sphinxbfcode{\sphinxupquote{createWorkingFolder}}}{\emph{base='.'}}{}
Creates a folder to drop files using the next available name.

\end{fulllineitems}

\index{doRoutes() (in module taipanPyRouter)}

\begin{fulllineitems}
\phantomsection\label{\detokenize{reference:taipanPyRouter.doRoutes}}\pysiglinewithargsret{\sphinxcode{\sphinxupquote{taipanPyRouter.}}\sphinxbfcode{\sphinxupquote{doRoutes}}}{\emph{args}}{}
\end{fulllineitems}

\index{findCrossingGroups() (in module taipanPyRouter)}

\begin{fulllineitems}
\phantomsection\label{\detokenize{reference:taipanPyRouter.findCrossingGroups}}\pysiglinewithargsret{\sphinxcode{\sphinxupquote{taipanPyRouter.}}\sphinxbfcode{\sphinxupquote{findCrossingGroups}}}{\emph{tickArray}, \emph{bugStatus}}{}
Identify crossing groups within the direct paths
\begin{quote}\begin{description}
\item[{Parameters}] \leavevmode
\sphinxstyleliteralstrong{\sphinxupquote{tickArray}} \textendash{} array with the starting and ending points

\item[{Returns}] \leavevmode
np.ndarray Collection of pairs of crossing bugs.

\item[{Return type}] \leavevmode
crossingBugs

\end{description}\end{quote}

\end{fulllineitems}

\index{initialiseTickArray() (in module taipanPyRouter)}

\begin{fulllineitems}
\phantomsection\label{\detokenize{reference:taipanPyRouter.initialiseTickArray}}\pysiglinewithargsret{\sphinxcode{\sphinxupquote{taipanPyRouter.}}\sphinxbfcode{\sphinxupquote{initialiseTickArray}}}{\emph{parkPos}, \emph{bugsTargetXY}}{}
Creates a new tick array with 2 only ticks (direct path)
\begin{quote}\begin{description}
\item[{Parameters}] \leavevmode\begin{itemize}
\item {} 
\sphinxstyleliteralstrong{\sphinxupquote{parkPos}} \textendash{} array with the starting points

\item {} 
\sphinxstyleliteralstrong{\sphinxupquote{bugsTargetXY}} \textendash{} array with the end points

\end{itemize}

\item[{Returns}] \leavevmode
3D np.array {[}lemoID-1, tick, coords{]}

\item[{Return type}] \leavevmode
tickArray

\end{description}\end{quote}

\end{fulllineitems}

\index{loadParkPosJSON() (in module taipanPyRouter)}

\begin{fulllineitems}
\phantomsection\label{\detokenize{reference:taipanPyRouter.loadParkPosJSON}}\pysiglinewithargsret{\sphinxcode{\sphinxupquote{taipanPyRouter.}}\sphinxbfcode{\sphinxupquote{loadParkPosJSON}}}{\emph{filename='locationProperties.json'}, \emph{folder='.'}}{}
Reads the park position file
\begin{quote}\begin{description}
\item[{Parameters}] \leavevmode\begin{itemize}
\item {} 
\sphinxstyleliteralstrong{\sphinxupquote{filename}} (\sphinxstyleliteralemphasis{\sphinxupquote{str}}) \textendash{} The name of the input file

\item {} 
\sphinxstyleliteralstrong{\sphinxupquote{folder}} (\sphinxstyleliteralemphasis{\sphinxupquote{str}}) \textendash{} The location of the input file

\end{itemize}

\item[{Returns}] \leavevmode
2D - {[}{[}x{]},{[}y{]}{]} parked positions coordinates indexed by LemoId-1
bugStatus (np.ndarray) : 1D {[}status{]} bugStatus indexed by LemoId-1
bugTypes (np.ndarray) : 1D {[}types{]} bugTypes indexed by LemoId-1

\item[{Return type}] \leavevmode
parkPos (np.ndarray)

\end{description}\end{quote}

\begin{sphinxadmonition}{note}{Note:}
Currently hardcoded array size to 309 bugs
This function should read from the database
\end{sphinxadmonition}

\end{fulllineitems}

\index{openS2JSONTile() (in module taipanPyRouter)}

\begin{fulllineitems}
\phantomsection\label{\detokenize{reference:taipanPyRouter.openS2JSONTile}}\pysiglinewithargsret{\sphinxcode{\sphinxupquote{taipanPyRouter.}}\sphinxbfcode{\sphinxupquote{openS2JSONTile}}}{\emph{fileName}, \emph{folder='./jsonTiles\_s2'}}{}
Reads the JSON file containing the target information (S2)
\begin{quote}\begin{description}
\item[{Parameters}] \leavevmode\begin{itemize}
\item {} 
\sphinxstyleliteralstrong{\sphinxupquote{filename}} (\sphinxstyleliteralemphasis{\sphinxupquote{str}}) \textendash{} The name of the input file

\item {} 
\sphinxstyleliteralstrong{\sphinxupquote{folder}} (\sphinxstyleliteralemphasis{\sphinxupquote{str}}) \textendash{} The location of the input file

\end{itemize}

\item[{Returns}] \leavevmode
Bugs requested position {[}{[}x,y{]}{]} indexed by LemoId-1

\item[{Return type}] \leavevmode
bugsXY (np.ndarray)

\end{description}\end{quote}

\begin{sphinxadmonition}{note}{Note:}\begin{itemize}
\item {} 
Currently hardcoded array size to 309,2

\end{itemize}
\end{sphinxadmonition}

\end{fulllineitems}

\index{optimiseAllocation() (in module taipanPyRouter)}

\begin{fulllineitems}
\phantomsection\label{\detokenize{reference:taipanPyRouter.optimiseAllocation}}\pysiglinewithargsret{\sphinxcode{\sphinxupquote{taipanPyRouter.}}\sphinxbfcode{\sphinxupquote{optimiseAllocation}}}{\emph{parkPos}, \emph{bugStatus}, \emph{bugTypes}, \emph{bugsTargetXY}, \emph{bugTargetTypes}}{}
Minimise the cost matrix of distances between a set of parked positions and set of targets.
\begin{itemize}
\item {} 
It assigns the best target positions based on minimum combined distance

\item {} 
Only the positions in parkPos that can be allocated have actual values, the rest of the values are NaNs.

\item {} 
This process allows the code to segment the allocation by fibre type

\end{itemize}
\begin{quote}\begin{description}
\item[{Parameters}] \leavevmode\begin{itemize}
\item {} 
\sphinxstyleliteralstrong{\sphinxupquote{parkPos}} (\sphinxstyleliteralemphasis{\sphinxupquote{np.ndarray}}) \textendash{} Park position for all starbugs

\item {} 
\sphinxstyleliteralstrong{\sphinxupquote{bugsTargetXY}} (\sphinxstyleliteralemphasis{\sphinxupquote{np.ndarray}}) \textendash{} Target positions to be allocated

\end{itemize}

\item[{Returns}] \leavevmode
Array of same shape of parkPos with the new allocations

\item[{Return type}] \leavevmode
newTargetsAlloc (np.ndarray)

\end{description}\end{quote}

\end{fulllineitems}

\index{shiftTickArray() (in module taipanPyRouter)}

\begin{fulllineitems}
\phantomsection\label{\detokenize{reference:taipanPyRouter.shiftTickArray}}\pysiglinewithargsret{\sphinxcode{\sphinxupquote{taipanPyRouter.}}\sphinxbfcode{\sphinxupquote{shiftTickArray}}}{\emph{movSeq}, \emph{tickArray}}{}
Given a sequence of bugIds, it creates the ticks to move bugs sequentially
\begin{itemize}
\item {} 
It reshapes tickArray as needed. {[}nBugs, len(movSeq)+1, 2{]}

\item {} 
It cascades the motion of the corresponding bugs sequentially as per movSeq order

\item {} 
It completes preceding ticks with initial tick pos for movSeq bugs

\item {} 
It adds target position to all tick following motion

\end{itemize}
\begin{quote}\begin{description}
\item[{Parameters}] \leavevmode\begin{itemize}
\item {} 
\sphinxstyleliteralstrong{\sphinxupquote{movSeq}} (\sphinxstyleliteralemphasis{\sphinxupquote{np.ndarray}}) \textendash{} List of bugIds to be moved in sequential order.

\item {} 
\sphinxstyleliteralstrong{\sphinxupquote{tickArray}} (\sphinxstyleliteralemphasis{\sphinxupquote{np.ndarray}}) \textendash{} Tick array to be appended with sequential motion

\end{itemize}

\item[{Returns}] \leavevmode
Shifted TickArray.

\item[{Return type}] \leavevmode
np.ndarray

\end{description}\end{quote}

\end{fulllineitems}

\index{writeOuputFile() (in module taipanPyRouter)}

\begin{fulllineitems}
\phantomsection\label{\detokenize{reference:taipanPyRouter.writeOuputFile}}\pysiglinewithargsret{\sphinxcode{\sphinxupquote{taipanPyRouter.}}\sphinxbfcode{\sphinxupquote{writeOuputFile}}}{\emph{S2FileName}, \emph{S3FileName}, \emph{tickArray}}{}
Writes an RTile (S3) from a tickArray
\begin{quote}\begin{description}
\item[{Parameters}] \leavevmode\begin{itemize}
\item {} 
\sphinxstyleliteralstrong{\sphinxupquote{S2FileName}} (\sphinxstyleliteralemphasis{\sphinxupquote{string}}) \textendash{} Input XYTile

\item {} 
\sphinxstyleliteralstrong{\sphinxupquote{S3FileName}} (\sphinxstyleliteralemphasis{\sphinxupquote{string}}) \textendash{} Output RTile

\item {} 
\sphinxstyleliteralstrong{\sphinxupquote{tickArray}} (\sphinxstyleliteralemphasis{\sphinxupquote{np.ndarray}}) \textendash{} routing ticks array

\end{itemize}

\end{description}\end{quote}

\end{fulllineitems}

\begin{itemize}
\item {} 
\DUrole{xref,std,std-ref}{genindex}

\end{itemize}


\renewcommand{\indexname}{Python Module Index}
\begin{sphinxtheindex}
\def\bigletter#1{{\Large\sffamily#1}\nopagebreak\vspace{1mm}}
\bigletter{t}
\item {\sphinxstyleindexentry{taipanPyRouter}}\sphinxstyleindexpageref{reference:\detokenize{module-taipanPyRouter}}
\end{sphinxtheindex}

\renewcommand{\indexname}{Index}
\printindex
\end{document}